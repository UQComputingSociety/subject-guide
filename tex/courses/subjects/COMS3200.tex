
\courseTemplate[
code = {COMS3200},
title = {Computer Networks I},
score = 4,
prereq = {COMP2303 OR CSSE2310},
contact = {3L, 1T, 1P},
coordinator = {Prof Neil Bergmann (bergmann@itee.uq.edu.au)},
assessment = {
Assignment 1 & 15\% & Theory questions and programming task \\
Assignment 2 & 15\% & Theory questions and programming task \\
Assignment 3 & 15\% & Theory questions and programming task \\
Final Exam & 55\% & Open book exam with multi-choice and short answer questions. \\
},
review = {
    This course provides a thorough overview of modern computer networking, based around the TCP/IP stack. Much of the content will be familiar to students who have completed CSSE2310, but the theory is covered in greater detail here.\\
    
    The course lectures can be boring but do cover the content well. It is worth either attending the lectures or studying the lecture notes, but the course does not require much external study beyond that. Tutorials are pointless generally, but students are required to demo the programming components of their assignments in them.\\
    
    The assignments are all quite easy and can be completed with minimal effort. Nominally, all three assignments involve a theory and programming part, but in 2016 the last assignment had no programming component. For the Sprogramming parts, there is no restriction on choice of programming language.\\
    
    The final exam is open book, and is incredibly easy (based on the 2016 course). If you bring in a good set of notes and have a reasonable understanding the content you will have little trouble with the final. Past exams are reasonably representative, but some years have extra cryptography content not included in more recent years.\\
    
    If you want to get a better understanding of networking and/or want an easy course that requires minimal study beyond contact hours, this course is recommended.
},
preparation = {
    \item Read back over CSSE2310 networking notes
}]{}
