
\courseTemplate[
code = {DECO1800},
title = {Design Computing Studio I},
score = {4},
prereq = {DECO1400},
contact = {1L, 4C},
coordinator = {Ms Lorna Macdonald (lorna@itee.uq.edu.au)},
assessment = {
Project Part A & 15\% & A document outlining the concept and design \\
Project Part B & 15\% & Minimal implementation of the project and a document outlining changes made and group planning \\
Project Part C & 30\% & Final design and a document outlining final design decisions and reflection \\
Design Portfolio & 20\% & A full HTML website explaining the project process, the portfolio process, and a course reflection \\
Reflective Journal & 20\% & A weekly blog of progress made and lessons learned in the course \\
},
review = {
    Contrary to what you may think, DECO1800 is more of a course which teaches how to work in groups, rather than a course on web design skills. You will be placed into a group of 4 and you will be required to create a website which makes use of the Trove database. Unfortunately, group work projects are a bit of a 'dice roll' in who you are going to get, but this course aims to teach you how to work within groups to create a quality project. In some ways this course is like ENGG2800, but for IT students. The group work emphasis is made evident with the constant documentation and blogging of lessons you have learnt, and suggestions to improve the design process. \\

    The project will be divided into three parts: The proposal document, the minimum viable product, and the final product. A lot of the mark comes from the accompanying documents, so be prepared to write a lot of words. \\

    You will also learn how to present your work to a public audience using a mock portfolio. This will involve documenting your design process, and a reflection on the course itself. You may use a bootstrap, but templates are not allowed. \\

    A reflective journal will be required to document the whole project process. It expected that the journal will be updated weekly. Many students loathe the reflective journal, but that is mainly because they underestimate how much work is required to catch up after a month of no updates to their blog. \\

    While this course can be a nightmare for those who don't get a good group, or who don't know how to work within a group, it is essential for any developer or designer to know the art of compromise and conflict resolution. This course is especially important for aspiring user experience designs who want to build a solid portfolio for employers.
},
preparation = {
    \item The more meetings you organise with your group, the higher your grades will be. I will repeat that: THE MORE MEETINGS YOU ORGANISE WITH YOUR GROUP, THE HIGHER YOUR GRADES WILL BE.
    \item The reflective journal may seem like a pain, but those who consistently post quality content twice a week will easily get 100\% for the journal.
    \item Don't freak out too much over the portfolio. You don't need to start right away. If you have a good journal, you will not need to create much content for your portfolio. That being said, make sure you leave at least a week or two to design and implement the website.
}]{}
