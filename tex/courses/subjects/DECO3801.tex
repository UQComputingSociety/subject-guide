
\courseTemplate[
code = {DECO3801},
title = {Design Computing Studio 3},
score = {4},
prereq = {CSSE2002 or INFS2200},
contact = {1L, 1C},
coordinator = {Dr Alex Pudmenzky (a.pudmenzky@uq.edu.au)},
assessment = {
Individual PA & 5\% & Peer Critique. \\
Ongoing Communication & 5\% & Sending E-Mails to Clients. \\
Scrum and Git & 5\% & Keeping Trello up to date and using the given Git repo. \\
Team Charter & Pass/Fail & Agree to be nice to each other. \\
Return Brief & 5\% & Show that you understand the given brief. \\
Online Quiz & 5\% & Quiz on Scrum and Agile, unlimited attempts. \\
UX Profile & 10\% & Show designs of what you intend to build. \\
Minimum Viable Product & 10\% & Checkpoint with significant amount of development done. \\
Informative product \newline website & 20\% & Basic website for \newline marketing your project. \\
Final Product Demonstration and Documentation & 30\% & Tradeshow in the final week demonstrating your product and handover to client. \\
},
review = {
    This course serves as the second half of DECO3800, which does not need to be taken before in first semester, and is the build portion of the yearlong capstone design studio. Students take proposed designs from DECO3800 and continue work for real clients in producing a software application from a range of fields.

    The course requires the use of Trello, implementing Scrum, and using a provided git repository, associating marks with their continuous use. This course includes students from Info. Tech and Multimedia cohorts and encourages working together, mimicking a real world software team.

    The main project is divided into a few deliverables which serve as add-ons to continuous work on a chosen main project. Deliverables range from design and development to report writing and maintaining the Trello in 2016 there was an MVP that had to be produced early on as well as a user experience report.

    In 2016 the IPW website was worth 20\% of the overall grade however seemed to be misplaced with the 10\% UX Profile, with the latter requiring much more work to complete.

    This course typically has clients from a range of different backgrounds and students can choose what type of software they would like to produce. In the past there have been opportunies to work on projects that include machine learning, web development, embedded systems design and even games production. It is important to work out what type of software system you might work best with before or while attending the first lecture of the semester to meet clients.

    The course has lectures that attempt to deliver content relevant to producing the assessable deliverables, such as reports and a minimum viable product, however they will not cover any material needed in the actual production of your application. Contacts must be attended however as attendance is taken and can be used later in peer assessment.

    The course has an ongoing emphasis on the use of Scrum and Agile methodologies and requires students to make use of a Trello board to keep track of tickets and try to work on small features in 2 to 4 week periods. This typically falls down when you reach the halfway point when other assignments get in the way but it's best to at least try and keep track of what sprints you were supposedly doing for marks.

    There is potential at the end of the semester to actually hand over the software to your client and they may engage you to continue working on the product. The idea with the course is to connect final year students with potential employers however in the past only a few projects will be taken further.
},
preparation = {
    \item Look over the final presentation recordings from DECO3800 to see which project would most interest you.
}]{}
