
\courseTemplate[
code = {DECO3500},
title = {Social \&amp; Mobile Computing},
score = 2,
prereq = {DECO2500},
contact = {2L, 2T},
coordinator = {Dr Andrew Dekker (a.dekker@uq.edu.au)},
assessment = {
Project Idea & 10\% & A poster explaining your concept \\
Project Plan & 20\% & A document explaining the problem space and the concept \\
Project Implementation & 15\% & A prototype which demonstrates key interactions of the product \\
Project Standup & 15\% & Three fortnightly speeches of the progress made on the project worth 5\% each  \\
Promotional Material & 10\% & A poster, video or website explaining the concept to a wider audience \\
Reflection Essay & 30\% & An essay reflecting on the project and how it links to the lecture topics \\
},
review = {
    'Social \&amp; Mobile Computing' is a course designed to practially apply the skills learnt in DECO2500 to a real project. This involves defining a problem in journalism, and proposing a solution by building a digital prototype. It is required that students will perform usability tests and iterate on their designs from the feedback. \\

    Groups will be formed based on individual posters made on projects ideas. Then, you will work within these groups to produce a concept document within GitHub, a promotional poster or video of the concept, and a high-fidelity prototype. \\

    Unfortunately, this is a very do-it-yourself kind of course. You will be required to construct a plan to carry out the project and the various steps in the design process. Since there is a lot of self-paced group work, don't be fooled into thinking this is an 'easy 7' course, as there is a lot of room for error. There are three 'Standups', where each group talks to the tutors about their current progress and what work is to be completed.\\

    A large portion of the mark comes from the reflective essay done after the project. This essay outlines your design process, what you learnt when carrying out the design process, and what to do next time. This gives students a taste of constructing a basic research paper, as links to research in the field are required in the document.\\

    While the project work is great for any design portfolio, students from this course tend to be left wondering how the lecture material relates to the project work, and how what they learnt in this course can be applied to their own work. If you want to pursue a career in HCI or User Experience Design, then you will find some use for the course material in your career.
},
preparation = {
    \item Unfortunately, there is little help from the criteria sheets, and little feedback will be given until the work is marked. Therefore it is essential to ask the tutors and lecturer for clarification on the assessment to do well in this course.
    \item Organise as many meetings as possible with your group to in order to be up to date with the progress made and what work is to be done. The more meetings you organise, the higher your grades will be.
    \item Pay at least some attention to the weekly lecture topics. This will be essential to finding research for your reflection document.
}]{}
