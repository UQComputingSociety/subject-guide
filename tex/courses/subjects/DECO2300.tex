
\courseTemplate[
code = {DECO2300},
title = {Digital Prototyping},
score = 5,
prereq = {DECO1100 + (CSSE1001 or DECO1400)},
contact = {1L, 4P},
coordinator = {Ms Lorna Macdonald (lorna@itee.uq.edu.au)},
assessment = {
 Video Prototype & 18.75\% & A short video explaining the concept for your prototype, and a statement of purpose. \\
 Prototype 1 & 18.75\% & A digital prototype made in Unity, and a SOP \\
 Prototype 2 & 18.75\% & A physical prototype made with the MakeyMakey, and a SOP \\
 Prototype 3 & 18.75\% & An iteration of the prototype which explores the prototype beyond the original concept, and a SOP \\
 Blog & 25\% & A weekly blog consisting of responses to contact classes, and an explanation of the process of creating your prototype. \\
},
review = {
    This course is all about prototyping designs. This will involve using the MakeyMakey and Unity to create new ideas for interfaces which will need testing and iteration.\\

    There is no group work in this course, nor is there a final exam. This is a very practical course. 75\% of the grade comes from the project which is split into four parts. Each part of the project will require you to test the prototype with students inside your tute. The first part of the project is a 3-5 minute video which explains the fundamentals of your concept. The second part is a digital prototype made with Unity and operated with a keyboard. The third part is the same digital prototype but with a physical interface. The fourth part is a major iteration on the prototype which helps explore the possibilities of the concept and the prototype. Each part has very similar criteria sheets, so doing well early can give you a good framework for the next parts. \\

    A statement of puropose is required for each part of the project. These generally range from 1500-3000 words. If you despise writing documentation, then this course is not for you.\\

    A weekly blog is required to respond to the contact class questions and to keep a log of your process. As with DECO1800, The blog can either your best friend or your worst enemy. If you consistently post around twice a week, it shouldn't be too hard to get 90\% or above for the blog.\\

    Overall, this is a relatively easy course compared to other courses due to the lack of group work or an exam, but it is quite an eye-opener to see how prototyping and testing is to improving a product.\\
},
preparation = {
    \item You will be given three 'topics' to choose from for the prototype. If there is an option to do a 'Game Mashup', it is highly recommended to choose that. Students that chose this topic tended to have a lot more depth and sense or purpose compared to the other topics.
    \item KEEP UP THE BLOG WEEKLY. It is very hard to remember what you did 2 weeks ago, so update the blog consistently. Be sure to take lots of pictures of the prototype to impress the tutors.
    \item Remember that these are only prototypes. You do not need a completely polished or well designed project. You just need enough to test the prototyping goals. Do not spend all your time implementing things not essential to the design.
}]{}
