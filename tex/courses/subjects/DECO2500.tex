
\courseTemplate[
code = {DECO2500},
title = {Human-Computer Interaction},
score = {4},
prereq = {DECO1400 or CSSE1001},
contact = {4C},
coordinator = {Prof Penelope Sanderson (psanderson@itee.uq.edu.au)},
assessment = {
Practical & 40\% & 10 weekly group work pracs worth 4\% each. \\
Report & 25\% & A 2000 word report from a choice of 20 topics \\
Final Exam & 35\% & A final exam on every chapter from the lectures and the textbook \\
},
review = {
    Human-Computer Interaction is a theory-oriented course. Many students come into the course expecting to do coding or design work and become sorely disappointed. \\

    Assessment involves weekly practicals, a research paper, and a final exam. 40\% of the grade are weekly pracs worth 4\% each. These weekly pracs can be easy marks if you prepare beforehand. As with any group work, be sure to find good people to work with in the first week. \\

    The second big assessment is the research document from a choice of 20 topics. Some involve real-world testing, and some are just literature reviews. Feel free to pick whatever you'd like, but some topics are easier than others. \\

    The final exam will quiz you on the textbook. Make sure to read it cover to cover, as you will be quizzed on specific case studies in the book. Past exam questions will be your friend when studying for the final exam. \\

    Penelope Sanderson is a lecturer from the Psychology department. If you approach the course as more of a psychology course than an engineering course, then you will have better expectation and understanding of the activities. Not everyone will enjoy this course, but the course gives a solid foundation to understanding how usabilitiy testing makes for better software.

},
preparation = {
    \item Prepare for the weeky practical classes beforehand. The criteria sheets for each week are released before the class begins, so no excuses.
    \item You will be required to find high-quality sources for your research paper. Using bibliography software such as Zotero makes this much easier. There are heavy penalties for going over the word count.
    \item Spread out your reading of the textbook over the semester, rather than at the last minute. For best results, find classmates who will quiz you on your knowledge. It is easier to understand the material than to memorise it.
}]{}
