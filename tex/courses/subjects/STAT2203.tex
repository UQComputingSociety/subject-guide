
\courseTemplate[
code = {STAT2203},
title = {Probability Models and Data Analysis for Engineering},
score = {3},
prereq = {MATH1051},
contact = {3L, 1T, 1P},
coordinator = {Dr Thomas Taimre (t.taimre@uq.edu.au)},
assessment = {
Assignment Problems & 48\% & 6 problem-based assignments \\
Mid-sem exam & 13\% & Problem-based exam \\
Final exam & 39\% & Problem-based exam \\
},
review = {
    This course teaches the basics of statistics and statistical analysis. Its content is very heavy on the maths and less so on practical application, which may be a positive or negative depending on the student. \\
    
    Throughout the course, you will learn about basic probability theory, probability distributions and hypothesis testing. The assignments usually have some questions which are answered by hand, and others that require coding (the course primarily uses MATLAB; in some years it has allowed students to use whichever language suits them) to develop and run statistical simulations.\\
	The problems on the final examination have in the past been similar to the assignments, but often of higher difficulty, requiring synthesis of multiple concepts.\\
	
	Overall, the course content is very useful, as a working knowledge of basic statistics is required for many higher level courses (for example, AI and Machine Learning). However, students may find the content of this course quite `dry', and the assessment difficult.
},
preparation = {
    \item Read up on basic probability and statistics
    
    \item Begin the assignments early; problems often take several days to fully `get'
    
    \item The lectures are fully recorded with complete slides uploaded, and are vital for revision material
}]{}
