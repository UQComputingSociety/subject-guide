
\courseTemplate[
code = {COMP4403},
title = {Compilers and Interpreters},
score = {5},
prereq = {COMP2502 or COMP3506},
contact = {3L, 1T},
coordinator = {Prof Ian Hayes (ianh@itee.uq.edu.au)},
assessment = {
Assignment 1 & 20\% & Add features to a Recursive Descent PL0 compiler. \\
Assignment 2 & 20\% & Add features to a Parser Generator-based PL0 compiler. \\
Assignment 3 & 20\% & Add features to a Parser Generator-based PL0 compiler. \\
Final Exam & 40\% & Closed book (one A4 sheet allowed). Examines all theoretical course content. \\
},
review = {
    The course covers basic compiler design. It splits the course up into theoretical (taught in the lectures and examined on the final) and practical content (through the assignments).
    
    In the theoretical component, it covers context-free grammars, recursive descent and bottom up LR(1) / LALR(1) parsing, static checking, code generation, garbage collection and object-orientation. The practical component consists of three assignments, where students are given the Java source code for a compiler for the PL0 programming language and are to add additional features to the language.
    
    The textbook is not required, but attending the lectures is highly recommended as the content, while not too difficult to grasp, does require explanation, plus Professor Hayes is a fantastic lecturer. Professor Hayes also provides comprehensive PDF notes of all topics based on the lectures that make excellent study tools.
    
    The final exam is closed book, but one double-sided A4 sheet of paper is allowed. Preparing for the exam is easy due to the aforementioned lecture notes, tutorial questions and numerous past exams available. The structure of the course has remained static for a number of years, so previous exams provide a good guide as to what to expect on the final.
    
    If you are interested in how programming languages work, I highly recommend this course. It's one of the best organised and taught courses in the school that I have personally taken, the content is interesting, and the assessment structure makes high grades achievable with moderate effort.
},
preparation = {
    \item Complete CSSE2010 and COMP3506, and in particular understand assembly code, state machines and tree data structures
    \item Do background reading on formal language theory (e.g. formal grammars, the Chomsky Heirarchy)
}]{}
