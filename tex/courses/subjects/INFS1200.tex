
\courseTemplate[
code = {INFS1200},
title = {Introduction to Information Systems},
score = {3},
prereq = {None},
contact = {3L, 1T, 1P},
coordinator = {Prof Xue Li (xueli@itee.uq.edu.au)},
assessment = {
    Assignment 1 & 10\% & ER Modelling and Mapping \\
    Assignment 2 & 10\% & Dabase System Implementation \\
    Onine Quizzes & 5\% & Online Quizzes Based on Practical Exercises \\
    In Class Quiz 1 & 15\% & Based on Theory Half of Content \\
    In Class Quiz 2 & 10\% & All SQL Query Problems \\
    Final Exam & 50\% & Covers Entire Content of Course \\
},
review = {
    This course is UQ's introduction to databases and SQL. The first half is very theory-heavy, taking you through the logic behind relational database structure. The theory is very straightforward and you definitely don't need to attend lectures (by the end of semester less than a third of students will be there).

    The second half of semester focuses primarily on SQL, with students doing the majority of their SQL work through phpMyAdmin. Assignments and exams are all quite simple, provided you've had a brief read over the course notes. If you've never used SQL before the LDBM tool will be your most valuable resource in preparing for Quiz 2 and the final exam. Functional Dependency is the most difficult topic in the course, but there are plenty of resources provided to help you understand it.

    Overall the course isn't terrible, just terribly boring. Save yourself the trauma of lectures and attend practicals and tutorials instead, they're far more helpful. 
},
preparation = {
    Do not buy the textbook. Download MySQL and a test database to play around with if you're incredibly keen.
}]{}
