\newlength\mylen
\newlength\mylena
% Icons for each section
\newarray\sectionicon
\readarray{sectionicon}{
  \faHome & % 1
  \faList & % 2
  \faUsers & % 3
  \faCode & % 4
  \faTree & % 5
  \faBook & % 6
  \faDatabase & % 7
  \faBatteryQuarter & % 8
  \faUsb & % 9
  \faWifi & % 10
  \faAreaChart & % 11
  \faBarChart & % 12
  \faTerminal & % 13
  Ai & % 14
  \faShield & % 15
  \faSitemap & % 16
  \faServer & % 17
  \faHandSpockO & % 18
  \faUserPlus & % 19
  \faQuestion & % 20
}
% Colours for each section
\definecolor{sectioncolor1}{HTML}{FFFFFF}
\definecolor{sectioncolor2}{HTML}{7B34BC}
\definecolor{sectioncolor3}{HTML}{3498DB}
\definecolor{sectioncolor4}{HTML}{3EC43B}
\definecolor{sectioncolor5}{HTML}{C4C154}
\definecolor{sectioncolor6}{HTML}{CF6193}
\definecolor{sectioncolor7}{HTML}{107530}
\definecolor{sectioncolor8}{HTML}{CF2B28}
\definecolor{sectioncolor9}{HTML}{CFAD2B}
\definecolor{sectioncolor10}{HTML}{CF3B21}
\definecolor{sectioncolor11}{HTML}{F7DD38}
\definecolor{sectioncolor12}{HTML}{CE0D00}
\definecolor{sectioncolor13}{HTML}{5B7CC4}
\definecolor{sectioncolor14}{HTML}{04CACE}
\definecolor{sectioncolor15}{HTML}{575CCE}
\definecolor{sectioncolor16}{HTML}{EF7F02}
\definecolor{sectioncolor17}{HTML}{BC8558}
\definecolor{sectioncolor18}{HTML}{7ABC67}
\definecolor{sectioncolor19}{HTML}{CF316D}
\definecolor{sectioncolor20}{HTML}{A3BCA7}

\def\sectioncolour{{"sectioncolor1", "sectioncolor2", "sectioncolor3", "sectioncolor4", "sectioncolor5",
"sectioncolor6", "sectioncolor7", "sectioncolor8", "sectioncolor9", "sectioncolor10", "sectioncolor11",
"sectioncolor12", "sectioncolor13", "sectioncolor14", "sectioncolor15", "sectioncolor16", "sectioncolor17",
"sectioncolor18", "sectioncolor19", "sectioncolor20"}}

\def\sectionlink{{"core:about", "core:lists", "cat:team", "cat:programming", "cat:algo",
"cat:theory", "cat:database", "cat:elec", "cat:embedded", "cat:network", "cat:math",
"cat:stat", "cat:os", "cat:ai", "cat:security", "cat:web", "cat:hp",
"cat:hci", "cat:management", "cat:?"}}
\tikzset{
  tab/.style={
    text width=30mm,
    draw=gray,
    thick,
    rectangle,
    align=center,
    inner sep=0pt,
    fill=gray!20,
    font=\sffamily\LARGE,
    color=black
  }
}

\AtBeginDocument{
  % calculation of the width for each tab
  \setlength\mylen{\paperheight}
  \ifnum\totvalue{section}>0
    \setlength\mylena{\dimexpr\mylen/\totvalue{section}\relax}
  \fi

  \backgroundsetup{
    scale=1,
    color=black,
    angle=0,
    opacity=1,
    contents= {
    \begin{tikzpicture}[
      remember picture,
      overlay,
    ]
    \ifnum\thesection>0
      \ifnum\totvalue{section}>0
        \node[inner sep=0pt]
            at (current page.center) (border) {
              \rule{0pt}{\dimexpr\textheight+2cm\relax}};
          \foreach \valsection in {0,...,\numexpr\totvalue{section}-1\relax} {%
            \pgfmathsetmacro{\sc}{\sectioncolour[\valsection]}
            \pgfmathsetmacro{\sl}{\sectionlink[\valsection]}
            \node[tab,minimum height=\mylena, fill=\sc, draw=\sc]
                at ([yshift=-(0.5+\valsection)*\mylena]current page.north west)
                (tab-\valsection)
              {\hspace*{10mm}
                 {\hyperlink{\sl}{\sectionicon(\numexpr\valsection+1)}}};
          }
        \pgfmathsetmacro{\sc}{\sectioncolour[\numexpr\thesection-1]}
        \node[
          draw=\sc,
          line width=2pt,
          rectangle,
          inner sep=0pt,
          text width=6mm,
          fill=\sc
        ] at ([xshift=18mm]current page.west){\rule{0pt}{\paperheight}};
        \node[font=\LARGE\sffamily,fill=white] at (border.south){\makebox[3em][c]{\thepage}};
      \fi
    \fi
    \end{tikzpicture}}
  }
}